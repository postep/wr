\documentclass{article}
\usepackage[T1]{polski}
\usepackage[polish]{babel}
\usepackage[utf8]{inputenc}
\usepackage[T1]{fontenc}
\usepackage[mediumspace,mediumqspace,Grey,squaren]{SIunits}
\usepackage{graphicx}
\usepackage{filecontents}
\usepackage{listings}



\graphicspath{ {./images/} }

\begin{document}
\title{Kompilacja skrośna dla systemu Lego Mindstorm}   
\author{Jakub Arnold Postępski} 
\date{\today}

\maketitle

\begin{abstract}
	Celem jest przyspieszenie procesu kompilacji dla kostek Lego Mindstorm w laboratorium robotyki. 
\end{abstract}

\section{Wstęp}
	Kostki Lego Mindstorm EV3 stosowane w laboratorium mają procesor z rdzeniem ARM926EJ-S (architektura ARM, taktowanie 300 MHz, 64 MB RAM). Stosowany system operacyjny ev3dev (oparty na Debian Wheezy). Komunikacja z kostkami przez SSH i Wi-Fi. System operacyjny zestawu instalowany jest na karcie SD.\\
	Konfiguacja sprzętowa spowalnia proces użytkowania całego zestawu. Niewydajny proces kompilowania programów w języku C++. Wolne działanie interpretera języka Python. Wolne działanie instalatora oprogramowania apt.\\
	Przy pomocy środowiska Docker opisano:
	\begin{itemize}
		\item metodę przygotowania gotowego OS z już zainstalowanymi pakietami
		\item proces kompilacji skrośnej
	\end{itemize}
	Opisano proces kopiowania gotowego kontenera na kartę SD przy pomocy programu Brickman.
	
\section{Docker}
Docker to środowisko do obsługi kontenerów podobnych w działaniu do wirtualnych maszyn. Możliwe uruchamianie kontenerów z innymi architekturami niż system maszyna gospodarza. Demon dla systemu operacyjnego. Możliowść tworzenia kontenera, zachowania zmodyfikowanego kontenera oraz pobrania gotowego kontenera z dedykowanych repozytoriów. Do raz uruchomionego kontenera można zalogować się wiele razy. Uruchomiony kontener posiada własną etykietę. Obraz może być uruchomiony wiele razy jako różne kontenery.

\section{Operacje na kontenerze Dockera}
Dla systemu z rodziny Ubuntu. Polecenia dla użytkownika z odpowiednimi uprawnieniami.

\subsection{Instalacja}

\begin{itemize}
	\item Instalacja: 
	\begin{lstlisting}
	apt install docker.io
	\end{lstlisting}
	\item Uruchomienie demona: 
	\begin{lstlisting}
	service docker.io
	\end{lstlisting}
\end{itemize}

\subsection{Uruchomienie kontenera}
Domyślnie użytkownik: robot, hasło: master
\begin{itemize}
	\item Ściągnięcie obrazu kontenera: 
	\begin{lstlisting}
	docker pull ev3dev/ev3dev-jessie-ev3-generic
	\end{lstlisting}
	\item Nadanie tagu kontenerowi: 
	\begin{lstlisting}
	docker tag ev3dev/ev3dev-jessie-ev3-generic ev3
	\end{lstlisting}
	\item Uruchomienie i logowanie do kontenera: 
	\begin{lstlisting}
	docker run --rm -it ev3 su -l robot
	\end{lstlisting}
\end{itemize}

\subsection{Zapis kontenera na karcie}
Należy poprzedzić ściągnieciem repozytorium
\begin{lstlisting}
git clone git://github.com/ev3dev/brickstrap
\end{lstlisting}
Skrypt programu w podkatalogu
\begin{lstlisting}
brickstrap/src
\end{lstlisting}
Dla działającego kontenera tworzy obraz karty SD:
\begin{lstlisting}
./brickstrap.sh create-tar ev3ros ev3ros.tar
./brickstrap.sh create-image ev3ros.tar ev3ros.img
\end{lstlisting}

Sprawdzenie którym urządzniem jest karta SD:
\begin{lstlisting}
fdisk -l
\end{lstlisting}

Zapis obrazu na kartę SD:
\begin{lstlisting}
dd if=ev3ros.img of=/dev/sdb bs=4M
\end{lstlisting}


\end{document}