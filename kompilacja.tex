\documentclass{article}
\usepackage[T1]{polski}
\usepackage[polish]{babel}
\usepackage[utf8]{inputenc}
\usepackage[T1]{fontenc}
\usepackage[mediumspace,mediumqspace,Grey,squaren]{SIunits}
\usepackage{graphicx}
\usepackage{filecontents}
\usepackage{listings}
\usepackage{hyperref}
\usepackage{url}




\graphicspath{ {./images/} }

\begin{document}
\title{Praca z kontenerami Dockera dla Lego Mindstorms EV3\cite{lego}}   
\author{Jakub Arnold Postępski} 
\date{\today}

\maketitle

\section{Wstęp}
Kostki Lego Mindstorms EV3\cite{lego} stosowane w laboratorium mają procesor z rdzeniem ARM926EJ-S (architektura ARM, taktowanie 300 MHz, 64 MB RAM). Stosowany system operacyjny ev3dev (oparty na Debian Wheezy). Komunikacja z kostkami przez SSH i Wi-Fi. System operacyjny zestawu instalowany jest na karcie SD.\\
Konfiguacja sprzętowa spowalnia proces użytkowania całego zestawu. Niewydajny proces kompilowania programów w języku C++. Wolne działanie interpretera języka Python. Wolne działanie instalatora oprogramowania apt.\\
Przy pomocy środowiska Docker\cite{docker} opisano:
\begin{itemize}
\item metodę przygotowania gotowego OS z już zainstalowanymi pakietami
\item proces kompilacji skrośnej
\end{itemize}
Kopiowanie gotowego kontenera na kartę SD przy pomocy programu Brickstrap\cite{brickstrap}.
	
\section{Docker}
Docker\cite{docker} to środowisko do obsługi kontenerów podobnych w działaniu do wirtualnych maszyn. Możliwe uruchamianie kontenerów z innymi architekturami niż system maszyna gospodarza. Demon dla systemu operacyjnego. Możliowść tworzenia kontenera, zachowania zmodyfikowanego kontenera oraz pobrania gotowego kontenera z dedykowanych repozytoriów. Uruchomiony kontener posiada własną etykietę. Obraz może być uruchomiony wiele razy jako różne kontenery. Do jednego kontenera można uruchomić wiele terminali.

\section{Operacje na kontenerze Dockera}
Dla systemu z rodziny Ubuntu. Polecenia dla użytkownika z odpowiednimi uprawnieniami.\\
\subsection{Instalacja}
Instalacja: 
\begin{lstlisting}
apt install docker.io
\end{lstlisting}
Uruchomienie demona: 
\begin{lstlisting}
service docker.io
\end{lstlisting}


\subsection{Uruchomienie kontenera}
Ściągnięcie obrazu kontenera\cite{docker-library}\cite{ev3}: 
\begin{lstlisting}
docker pull ev3dev/ev3dev-jessie-ev3-generic
\end{lstlisting}
Nadanie tagu obrazowi : 
\begin{lstlisting}
docker tag ev3dev/ev3dev-jessie-ev3-generic ev3
\end{lstlisting}
Wyświetlenie dostępnych obrazów: 
\begin{lstlisting}
docker images
\end{lstlisting}
Uruchomienie i logowanie do kontenera: 
\begin{lstlisting}
docker run -it --name <nazwa_kontenera> ev3
\end{lstlisting}
Dodatkowo, w celu ułatwienia pracy można stosować poniższe:
\begin{itemize}
\item W nowej konsoli hosta, wyświetlenie uruchomionych kontenerów: 
\begin{lstlisting}
docker ps
\end{lstlisting}
\item Logowanie do już utworzonego kontenera: 
\begin{lstlisting}
docker exec -it <nazwa_kontenera> bash
\end{lstlisting}
\item Zabicie kontenera, z systemu hosta: 
\begin{lstlisting}
docker kill <nazwa_kontenera>
\end{lstlisting}
\end{itemize}

\subsection{Praca z kontenerem i kompilacja skrośna}
Po zalogowaniu do kontenera można wykonywać operacje jak dla maszyny wirtualnej. Aby nie stracić efektów pracy należy zapisać migawkę kontenera z systemu hosta.
\begin{lstlisting}
docker commit <tag_obrazu> <nazwa_kontenera> 
\end{lstlisting}
Powoduje to utworzenie nowego obrazu. Dzięki temu możemy wytworzyć dowolne modyfikacje systemu (takie jak instalacja oprogramowania) i potem przenieść na kartę SD.\\
Jeśli kontener zostanie uruchomiony poleceniem \textit{run} z parametrem \textit{-v} 
\begin{lstlisting}
docker run -it -v <katalog_hosta>:<katalog_kontenera> 
--name <nazwa_kontenera> <nazwa_obrazu>
\end{lstlisting}
to katalog o danej nazwie zostanie odtworzony w kontenerze. Dzięki temu mamy łatwą możliwość kopiowania plików pomiędzy kontenerem a maszyną hosta. Po zainstalowaniu w kontenerze:
\begin{lstlisting}
apt-get install unzip bzip2 build-essential
\end{lstlisting}
otrzymujemy możliwość kompilacji skrośnej.
Dla przykładowego programu, w kontenerze, we współdzielonym katalogu wykonujemy:
\begin{lstlisting}
g++ -o hello hello.c
\end{lstlisting}
\begin{lstlisting}
./hello
\end{lstlisting}
Następnie jesteśmy w stanie przesłać gotowy program na rzeczywistą kostkę, z systemu hosta. Po przejściu do katalogu współdzielonego:
\begin{lstlisting}
scp hello robot@192.168.18.74:~/hello
\end{lstlisting}


\subsection{Zapis kontenera na karcie}
Należy wykonywać w konsoli hosta oraz pamiętać aby wytworzyć nową migawkę obrazu przed zapisem na kartę\\
Ściągniecie repozytorium:
\begin{lstlisting}
git clone git://github.com/ev3dev/brickstrap
\end{lstlisting}
Skrypt programu w podkatalogu
\begin{lstlisting}
brickstrap/src
\end{lstlisting}
Dla kontenera o podanym tagu tworzy obraz karty SD\cite{brickstrap}:
\begin{lstlisting}
./brickstrap.sh create-tar <nazwa_obrazu_dockera> ev3.tar
./brickstrap.sh create-image ev3.tar ev3ros.img
\end{lstlisting}
Sprawdzenie którym urządzniem jest karta SD:
\begin{lstlisting}
fdisk -l
\end{lstlisting}
Zapis obrazu na kartę SD:
\begin{lstlisting}
dd if=ev3ros.img of=/dev/sdb bs=4M
\end{lstlisting}
Jeśli hasło nie zostało zmienione, po uruchomieniu kostki należy się zalogować użytkownik: \textit{robot}, hasło \textit{master}

\bibliographystyle{acm}
\bibliography{bibliography}

\end{document}