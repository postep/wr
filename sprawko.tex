\documentclass{article}
\usepackage[T1]{polski}
\usepackage[polish]{babel}
\usepackage[utf8]{inputenc}
\usepackage[T1]{fontenc}
\usepackage[mediumspace,mediumqspace,Grey,squaren]{SIunits}
\usepackage{graphicx}
\usepackage{filecontents}
\usepackage{listings}




\graphicspath{ {./images/} }

\begin{document}
\title{Sprawozdanie z Laboratorium Wstępu do Robotyki}   
\author{Jakub Arnold Postępski} 
\date{\today}

\maketitle

\section{Wstęp}
Celem projektu przebadanie możliwości instalacji klienta ROS Kinetic na kostce zestawu Lego Mindstorms EV3 \cite{lego}. Badania na prośbę prowadzących, pragnących zmienić zakres i cel laboratorium.

Studenci w trakcie laboratoriów budują określonego robota. Używają wgrywanego na kartę SD Debiana Wheezy. Wykorzysują bibliotekę Pythona do pisania skryptów sterujących robotem. W pesymistycznym przypadku używane są wszystkie porty zestawu. Połączenie z robotem przez Wi-Fi oraz SSH. Wymagana częstotliwość obsługi zdarzeń to 10 Hz.

Wolny procesor zestawu ARM926EJ-S (architektura ARM, taktowanie 300 MHz, 64 MB RAM) oraz system z karty SD (mała szybkość IO) utrudniają pracę oraz uniemożliwiają swobodną kompilację. 

\section{Konfiguracja kostki do pracy z ROS}
Niemożliwe uruchomienie tzw. mastera ROS ze względu na zasoby sprzętowe. Zastosowano podejście z wygenerowaniem nowego obrazu OS z już zainstalowanymi pakietami \cite{arnold}. Do kompilacji zainstalowano programem \textit{apt}:
\begin{lstlisting}
unzip 
bzip2
apt-utils
build-essential
cmake
initramfs-tools
libboost-all-dev
libboost-dev
libbz2-dev
libc6-dev
libconsole-bridge-dev
libgtest-dev
liblog4cxx10
liblog4cxx10-dev
liblz4-dev
libtinyxml-dev
libpython2.7-stdlib
libyaml-cpp-dev
libyaml-dev
python-coverage
python-empy
python-imaging
python-mock
python-netifaces
python-nose
python-numpy
python-paramiko
python-pip
python-yaml
\end{lstlisting}
i skompilowano ręcznie pakiet \textit{sbcl}. Zainstalowano pakiety przy pomocy programu \textit{pip}
\begin{lstlisting}
rosdep
rosinstall_generator
wstool
rosinstall
catkin_pkg
rospkg
\end{lstlisting}
Do pliku \textit{/etc/ros/rosdep/sources.list.d/20-default.list} dodano nowe repozytorium:
\begin{lstlisting}
yaml https://raw.githubusercontent.com/moriarty/ros-ev3/master/ev3dev.yaml
\end{lstlisting}
I skompilowano pakiet \textit{ros\_comm} przy wykorzystaniu \textit{catkin\_make\_isolated}
\begin{lstlisting}
'rosinstall_generator ros_comm common_msgs --rosdistro kinetic --deps 
--wet-only --tar > kinetic-ros_comm-wet.rosinstall'
wstool init src kinetic-ros_comm-wet.rosinstall
'./src/catkin/bin/catkin_make_isolated --install 
--install-space /opt/ros/kinetic -DCMAKE_BUILD_TYPE=Release'
\end{lstlisting}
\section{Testy}
Po zalogowaniu przez SSH obciążenie procesora na poziomie 5\% przez programy oraz 5\% przez OS. ROS Master uruchomiony na odpowiednio mocnej maszynie. Po uruchomieniu prostego skryptu Pythona wysyłającego wiadomość na określony temat z częstotliwością 10 Hz obciążenie procesora przez programy na poziomie 20\% oraz przez OS na poziomie 60\%. Wiadomości nadawane z właściwą częstotliwością (sprawdzenie po stronie odbiorcy, \textit{rostopic hz}). Po uruchomieniu drugiego procesu nadającego (z tą samą częstotliwością, inny temat) pełne obciążenie procesora (OS na poziomie 80\%) oraz obniżenie częstotliwości nadawania. Prosty skrypt z pętlą częstotliwości 10 Hz obsługujący dwa silniki oraz czujnik koloru i podczerwieni (bez obsługi ROSa) obciążał procesor na poziomie 50 \%.

\section{Wnioski}
Negatywny wynik badań. Brak odpowiedniej wydajności sprzętu przy wymaganej częstotliwości, nawet dla prostych programów, przez wysycenie procesora. Brak możliwości optymalizacji OS (niskie obciążenie procesora w stanie bezczynności). Brak możliwości optymalizacji przez przepisanie programów do C++ (biblioteka ROS obciąża procesor głównie przez odwołania systemowe).

\section{Zalecenia}
\begin{itemize}
	\item Kompilacja skrośna
	\item Napisanie serwera który będzie pośrednikiem między ROSem a urządzeniami zestawu i odpowiedniego klienta na mocniejszej maszynie
\end{itemize}

\bibliographystyle{acm}
\bibliography{bibliography}
\end{document}